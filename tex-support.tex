\documentclass[12pt]{article}
\usepackage[printonlyused,withpage]{acronym} % acronyms
\usepackage{amsmath}
\usepackage{amssymb}
\usepackage{amsthm}
\usepackage[page,title,toc]{appendix} % appendix
\usepackage{array} % tables (?)
\usepackage{bbold} % \mathbb
\usepackage[backend=biber, style=apa]{biblatex} % bibliography
\usepackage{bm} % bold math
\usepackage{caption} % better captioning and figure numbering (?)
\usepackage{changepage} % some landscape pages
\usepackage{csquotes} % \blockquote
\usepackage{enumitem} % control spacing
\usepackage{etoolbox} % for a macro (?)
\usepackage{fancyhdr}				% For fancy header/footer
\usepackage{float}      % use the [H] option for positioning
\usepackage[T1]{fontenc}
\usepackage{fontspec}
\usepackage[hang,flushmargin]{footmisc} % don't indent footnotes
\usepackage{gensymb} % \degree symbol
\usepackage{glossaries} % glossary
\usepackage{graphicx}				% For including figure/image
\usepackage[margin=1.0in,includehead,includefoot]{geometry}		% For setting margins
\usepackage{import}    % \incfig
\usepackage[iso,american]{isodate}
\usepackage{makecell}  % multi-row table cells
%\usepackage{mathptmx}  % make math Times New Roman
%\usepackage{minted}    % code styling
\usepackage{pdflscape} % make \incfig to landscape
\usepackage{pdfpages} % \includepdf
%\usepackage{pgffor} % iterate over the modular files to import their labels
\usepackage{pgfplots} % bar chart
\usepackage{ragged2e} % \raggedright
\usepackage{setspace} % {single,one-half}space \doublespacing
\usepackage{soul} % underline across line breaks
\usepackage{svg}
\usepackage{textcomp} % textcomp to get rid of gensymb warnings
%\usepackage{times} % Times New Roman, only pdflatex
\usepackage{transparent} % \incfig (breaks pgffor)
\usepackage[normalem]{ulem} % allow \smash of underlines
\usepackage{vcell} % vertical writing in table cells (?)
\usepackage{wrapfig} % \incfig
\usepackage{xifthen} % \incfig
\usepackage{xparse} % ?
\usepackage{xr} % ?

% hYpErReF mUsT cOmE lAsT
\usepackage[hidelinks]{hyperref} % \href{url}{text}
\hypersetup{colorlinks=true,linkcolor=black,urlcolor=blue,filecolor=black,citecolor=black}

% svg and Inkscape's pdf_tex
\newcommand{\incfig}[2]{
	\def\svgwidth{#1\columnwidth}
	\import{./}{#2.pdf_tex}
}

% Bibliography
%\addbibresource{}

% Font size reminder
% \tiny \small \normalsize \large \Large \Huge
%\setmainfont{Times New Roman}
\setmonofont{Mononoki Nerd Font Mono}[Scale=MatchLowercase]

%%% Macros
% Horizontal Spacing
\newcommand\hs{\hspace{1cm}}
\newcommand\hhs{\hspace{0.5cm}}
% Singlespacing in certain environments: \ssInEnv{environment, font=\normalsize}
\newcommand{\ssInEnv}[2][\normalsize]{\BeforeBeginEnvironment{#2}{\begin{singlespace*}{#1}}\AfterEndEnvironment{#2}{\end{singlespace*}}}
% Fancy header and footer for homeworks
\newcommand\hwheadfoot{
\pagestyle{fancy}
\fancyhead[LO,L]{Lorenzo Hess}
\fancyhead[CO,C]{CLASS: Homework NUMBER} % TODO
\fancyhead[RO,R]{\today} % TODO
%\fancyfoot[LO,L]{}
\fancyfoot[CO,C]{\thepage}
%\fancyfoot[RO,R]{}
\topmargin=-0.75in
\renewcommand{\headrulewidth}{0.4pt}
\renewcommand{\footrulewidth}{0.4pt}}

% Homework problems with arbitrary numbers without worrying about section formatting and counters
\newcommand{\problemSection}[1]{ % Args = section number
\noindent\Large\textbf{Section #1}
}
\newcommand{\problem}[1]{ % Args = number
\noindent\large\textbf{Problem #1}
}

% Volume symbol (heat transfer)
\newcommand{\volume}{{\ooalign{\hfil$V$\hfil\cr\kern0.08em--\hfil\cr}}}

% Table of Contents tweaks
% Singlespacing
\let\oldToC\tableofcontents
\renewcommand{\tableofcontents}{\begin{singlespace}\oldToC\end{singlespace}}
% Include subsubsections
\setcounter{tocdepth}{3}

% Spacing
\singlespacing
% \ssInEnv{itemize}
% \ssInEnv{enumerate}
% \ssInEnv{tabular}
\setlist{listparindent=\parindent, % indent paragraphs under \item
  nolistsep} % no spacing between list items

% Smash underlines
\let\oldunderline\underline
\renewcommand{\underline}[1]{\oldunderline{\smash{#1}}}
\setlength\ULdepth{1.5pt}

% Minted code styling
%\usemintedstyle{friendly}
%\setminted[octave]{fontsize=\scriptsize, breaklines=true, breakanywhere=true, linenos=true, numbersep=6pt, stripnl=true, baselinestretch=1}
%\setminted[text]{fontsize=\scriptsize, breaklines=true, breakanywhere=true, linenos=false, numbersep=6pt, stripnl=true, baselinestretch=1}

\author{Lorenzo Hess}
\date{\today}
\title{``TeX'' Support Proposal}

\begin{document}
\maketitle
\tableofcontents

\newpage
\section{Introduction}%
\label{intro}

This document outlines ideas for a comprehensive approach to \LaTeX\ advocacy, covering everything from pro-\LaTeX\ talking points, arguments against word processors (WPs), solutions for group work in \LaTeX, and \LaTeX\ code snippets, including full preambles and examples. While any individual could make use of this advocacy material, a group or club would be ideal in terms of providing additional resources and labor. Such a group could be called ``TeX Support'', so I've used that in the title of this document \footnote{Thanks to Keegan Kuhn for this pun.}.

\section{What's Our Agenda?}%
\label{agenda}

We like \LaTeX, and we think it's better for preparing documents, in most cases, than Word Processors (WPs). We're also pragmatic, and recognize that many people haven't used \LaTeX\ or may have felt frustrated when they have. Many \LaTeX\ tutorials indeed fail to enumerate and provide what we see as essential foundations for learning, appreciating, and enjoying \LaTeX\ document preparation. For example, many tutorials characterize \LaTeX's utility in terms of what it is good at, such as typesetting math or providing robust label and reference management. They fail, however, to provide a comprehensive feature comparison between \LaTeX\ and WPs that would make clear \LaTeX 's superiority. Tutorials also generally fail to fully compare the \LaTeX\ and WP mindsets, and thus fail to help users realize how WPs hinder writing. Finally, tutorials fail to provide essential tools that make \LaTeX\ editing less tedious, from code templates like preambles to software like Detexify.

We have two general goals: 1) present a comparative, serious, and evidence-based argument for when, and why, individuals and groups should use \LaTeX\ instead of WPs; and 2) to provide code templates, code snippets, and guides for individuals and groups to not only get started with \LaTeX, but use it to its full potential.

We're not here to evangelize. WPs do have certain advantages over \LaTeX (see the feature comparison in Section X), and it's a free country (and even freer Internet): we don't care what software you prefer or use. We do believe that \LaTeX, combined with a proper editor and editing workflow, usually makes writing and formatting faster and better.

\section{What is \LaTeX?}%
\label{what-is}

What is LaTeX? What is a markup language? Do I need to know programming? Is LaTeX programming?

You write instructions (i.e. code) to tell a latex compiler how to prepare your PDF, i.e. how to format, structure, and generally manipulate text, images, tables, etc to make your PDF.

Analogies: latex is the movable type of PDFs.

\section{Feature Comparison}%
\label{feat-comp}

\subsection{The General WP}%
\label{feat-comp.general-wp}

First cover features, operation (GUI: menus, buttons, and keybinds), and general workflow. Cover the WP mindset.

\underline{Advanced Features}

Most WPs lack advanced features such as generating a table of contents \footnote{As shown in the next section, many WPs can actually generate a table of contents. Their ToCs, however, are basic and impede workflow, and thus are not considered advanced.}, generating lists of figures, more, and more as well as configuration of all of the above.

\underline{A Hidden, Mutable State}

WPs are "What You See Is What You Get" \footnote{WYSIWYG, pronounced "Wi-see-wig"} editors. What appears in the WP window will appear in your PDF: what you see in the WP is what you're always gonna get. There's no separation between input and output: the writing you put into the WP will be present in the output, whether that's the WP file itself, a generated PDF, or a screenshot of the WP window, perhaps.

Because what you see is what you'll get, any characteristic of your document must be telescoped, projected, or mapped into visual form. Consider boldface styling. Boldface style is the thick version of a font; applying bold to a word makes its font thicker. Crucially, boldface must be applied /to a word/. A word processor, however, has no conception of a word. If you wish to have a word be in boldface, you can select the word by highlighting it and make it bold, often with a keybind or button. Your conception of the bold word, however, has not been mapped one-to-one to the WP: information has been lost. Specifically, your conception of the bold word relies on your conception of the word itself, and this conception in turn relies on the ability to define start and end points of a word. The WP, however, has no conception of a word, and thus cannot define its start and end points. Sure, you may have highlighted the word by starting at one letter and ending at another, but the WP doesn't know this. Your intention, of bolding that word itself (as defined by start and end points), has been lost in the projection of your conception of bolding onto the WPs visual space.

This information loss leads to weird, and often annoying, behaviors. If I bold a word, for example, and then place my cursor in the middle of the word and continue typing, the new letters will remain in bold. Even if I first type a space, which should, in my conception, signify the start of a new word, the letters are bold. In fact, WPs allow you to apply bold style to things that aren't even words, even to spaces. Hopefully my explanation has been clear enough, but consider how ludicrous it is to make a space boldface.

The many-to-one mapping from your conception to the visual WP environment doesn't simply cause the lost information to disappear. In fact, WPs require you to keep track of metadata about the document, which can be termed the document's "state".

% As a trivial example, consider a line break. In a WP, if you hit Ret twice, you'll make two new lines, and there'll be one line of separation between your current paragraph and the previous. While it may seem obvious, this line between the paragraphs is actually the visual telescoping, or projection, of a syntactical paragraph break onto the visual, WYSIWYG WP environment. In this case, the syntactical paragraph break can be mapped, one-to-one, from syntax-space to WP WYSIWYG space: the paragraph break is simply visual separation, and because the WYSIWYG editor provides a visual representation of your document, it can fully represent the visual paragraph break without any loss of information.

% For a similar example where information is lost by mapping into the WP's visual environment, consider the page break. Similar to the paragraph break, a syntactical page break provides separation from the end of the current paragraph up to the end of the page (the page break forces subsequent content to begin on the following page).

All this mapping into visual space requires you to keep track of state. You have to remember if you had put a page break somewhere, and if so, where on the page the break was inserted. Or if the boldface ends at the end of a boldface word or if it applies to the space after it as well.

"bold" emphasis style. When you emphasize a word with boldface, you desire that word to be bolded. The WP, however, has little, if any, conception or definition of a "word". In this visual environment

Because of the WP's limitations

%          3. Limits of Document Processors
%             1. Advanced features: table of contents, list of figures, configuration of all of the above.
%             2. Because you can't see what's going on, and the Document Processor reveals options depending on context (are you in a table, in a bold word, in an image), it can be difficult to memorize what's available. Contextual options often become associated with clicks, icons to click, menus, and in general chains of clicks and selections. As data structures, this essentially involves remembering a sequence of tasks which depend on the mutable state of the document. These mutable variables, such as the position of your cursor, what is highlighted, what is selected, etc, require cognitive effort to track, and thus require overhead that cannot be devoted to writing. LaTeX does require keeping track of things, but not mutable state; the commands, and related code, that you have to remember are immutable. Moreover, your document (e.g. PDF) is separate from your actual content (code). This separation creates less of a cognitive burden. Also, you know excactly what your content is: it's simply Unicode text, no invisible artifacts devoid of rhetorical meaning such as a "bold paragraphs" or. Because Document Processors don't define things from the bottom up in a way that the Document Preparation and Processing principles and functionalities are accessible to the user, they allow things like "bold paragraphs", headings that are only defined by font sizes, and page breaks that are inserted with C-Ret and aren't visibly signified by anything. Because DPPs don't offer a bottom-up way of building a document, they suffer from having to add features which result in typographically meaningless results, like a "bold paragraph". All this adds to state tracking ("I have to remember there's a page break here"). LaTeX is declarative. DPPs require you to memorize semantically lacking keybinds, and sequences of keybinds, to achieve typographic states that are never defined in the first place.

%                Consider this example where I use Ret and Del to move this paragraph. Or this other similar example. There's no documentation (verify this) on what state my paragraph is in, e.g. is it "twice indented", or "half indented", or "first-line indented", and there's no documentation on what keybinds to use to achieve these states. Google Drive forces you to memorize essentially arbitrary keybinds, and their sequences. We don't use the term "forces" to be dramatic: there's no other option, and these features are a direct result of their deliberate design and documentation choices.

%                What does it even mean to have a cursor next to an image in a DP? In LaTeX, figures are floats, not text.

%                Google Drive's ToC allows for different fonts. It allows you to actually change it, but those changes don't get reflected in the respective headings (not that LaTeX allows this-- it's just a bad design choice). It also allows comments, but the comments disappear when you generate the ToC. Its configuration is minimal: show page numbers, show tab leader (.....), and heading level -> indentation distance; not even changing separation of tab leader periods is allowed.

\subsection{\LaTeX}%
\label{feat-comp.latex}

Quoted from https://blog.orvium.io/latex-over-word/:
\begin{enumerate}
  \item LaTeX is specifically designed to produce high-quality typesetting, which makes your documents look professional and polished
  \item since you’re essentially writing code, you can meticulously fine-tune your document to look exactly the way you want it or in accordance with the highly specific requirements some journals have.
  \item LaTeX is optimized for minimal resource utilization. This allows researchers to work more efficiently on large documents with many equations, figures, images, and cross-references (think dissertations, books, or studies).
  \item LaTeX also generates a table of contents, a list of figures, and a complete list of references which you can manually edit in code
  \item the \\input and \\include commands allow you to split up sources in a controlled way, effectively making large documents into smaller files that can be managed separately.
  \item Since LaTeX operates with plain text files, the level of control you have as a user is beyond what traditional word processors can offer. This can prove very handy when collaborating with multiple authors on a big project, as it allows you to use tools like Git or SVN to implement version control and track changes.
  \item Anything that prominently features equations, tables, figures, or other designs is best completed via LaTeX.
  \item Academic dissertations and doctoral theses - from the reference system to the automatic and efficient table of contents, LaTeX makes working on gigantic projects such as these very easy. By comparison, researchers using Word frequently save chapters in separate documents to keep the software from lagging up or crashing and thereby losing their work.
\end{enumerate}

Regarding 1, go into what typsetting actually is, i.e. good-looking justification, text wrapping, math, equations.

Compare Overleaf opening ``10-page'' worth of latex code with google docs, onedrive 10-page document.\\

Quoted from https://www.baeldung.com/cs/latex-vs-word-main-differences:
\begin{enumerate}
  \item it serves as a broad category encompassing all word-processing software that immediately displays the final output. Having this characteristic, Word is usually called a What You See Is What You Get (WYSIWYG) editor rather than requiring initial file processing.
  \item On the other hand, LaTeX’s user interface is distinct from conventional word processors, as it relies on a markup language for document creation rather than a graphical interface
  \item Word supports real-time co-authoring, where multiple users can work on the same document simultaneously without conflicts.\\
\end{enumerate}

Quoted from https://tex.stackexchange.com/questions/110133/visual-comparison-between-latex-and-word-output-hyphenation-typesetting-ligat:
\begin{enumerate}
  \item how many unnecessary double paces or blank lines have any big Word document of an average user? This mistakes are hardly noticed and corrected and spoiled the format, but simply does not exist in LaTeX.
\end{enumerate}

Regarding 1, maybe Word can highlight two spaces to show an error, and you can just delete the extra space. What latex offers, however, is the ability to forget about these typos. You can focus on your writing, rather than these minute procedural details.\\

Quoted from https://nitens.org/w/latex/:
\begin{enumerate}
  \item Popular word processors either lack support for kerning tables or disable kerning by default
  \item Most word processors create fake small capitals by adjusting the size of capitals
  \item A good typesetting programme should always use contextual intelligence and substitution tables to determine whether ligatures are needed
        \item Readability results not only from a good selection of typefaces, but also from a correct distribution of characters and whitespace per line. To attain this goal, most WYSIWYG word processors use relatively dumb justification/hyphenation procedures
\end{enumerate}

Regarding all these, verify for newer WP versions (including cloud):

Quotes from https://www.andy-roberts.net/latex/benefits/:
\begin{enumerate}
  \item LaTeX is essentially a markup language. Content is written in plain text and can be annotated with various 'commands' that describe how certain elements should be displayed. The LaTeX interpreter reads in a LaTeX marked-up file, renders the content into a document and dumps it a new file. Therefore, it's not an interactive system that is the de-facto method for document creation nowadays.
  \item You can get LaTeX to do just about anything you can think of! Over the years, an overwhelming selection of packages to extend its potential and macros that can simplify complex tasks have come into being, most of which are freely available on CTAN
  \item LaTeX's main users are within academia and research institutions
  \item There are other crazy packages that you can install which allow you to typeset music scores, chessboards and cross-words! CTAN is the main repository of these resources. Most are well documented and as you can imagine, with LaTeX being around for so long, the number of extensions is vast. The chances are, if you're struggling to do a task, someone will have undoubtedly written a package to solve it easily!
  \item Even with simple documents, you can quickly become frustrated by Word's rather unintelligent interference. The hours that are wasted trying to position that image which you know will fit at the bottom of the page, but Word refuses to put it there! How many can relate to this experience? You have your 30 page document with text, tables and images. You just spent the evening getting it formatted nicely - all your figures in the right place and then you notice that one of your paragraphs isn't clear enough. You add one sentence, which then pushes an image on to the next page, leaving a massive gap at the bottom of that page where your image once was. This then daisy-chains down, knocking other tables and images out of place all the way to the end of your document! It's a real laugh. Fortunately, LaTeX is much more clever in this respect and positions your images and tables with a lot of common sense. So, if you want your image to appear at the bottom of a given page, it'll stay there!

        Whilst LaTeX makes decent typesetting decisions for you, if you want to, you can have total control over the presentation of your document.
  \item It's difficult to disagree that the output from LaTeX is far superior to what Word can produce. This is emphasised greatest when it comes to documents with high mathematical content, which is a major strength for LaTeX. It also has much better kerning, hyphenation and justification algorithms that simply make the output far more professional than what any word processor. Its algorithms for laying out text are more sophisticated and extremely fine-grained. For example, the accuracy is so high because it uses a measurement known as a scaled point which translates as 100th of the wavelength of natural light!

        LaTeX works with the concept of niceness (well, I suppose technically it's badness - which it works to minimise). LaTeX has a large set of metrics that it evaluates against when generating your document. It experiments with various permutations of parameters and determines the one which gives the "nicest" output. It can take the time to do this because it isn't interactive. Word processors don't have the computational resources available (yet) to carry out the equivalent calculations and still remain interactive. Also, many people forget that typesetting is actually a professional skill - people train for years to learn how to layout publications. Yet, as soon as you open a word processor, you go about committing typesetting sins all the way. Typesetters know for example that its easier to read sentences that are approximately 66 characters wide. Have a look in your books and count the letters! Also, why do newspapers and magazines have narrow columns? But, the default layout of a word processor gives an average of 100 words per line. I suppose many people don't mind, but you would notice if you read a lot of large documents.
  \item One of the reasons why perhaps so many people struggle with Word when creating large documents, is because it is prone to crashes. 'Document recovery' is now a high ranking feature of Word. I'm sure people would prefer if MS would just make their software more stable! ... Because LaTeX is so mature - and developed by extremely clever programmers - bugs are negligible. And even if it were buggy, then there is no risk of you ever losing your original source text. Where as with Word, almost any tool within its integrated environment is capable of corrupting your file if it causes a crash.
  \item Word may have the advantage of a GUI which is good for beginners. It reduces the cognitive load as it's a case of recognition verses recall.
  \item Word can be extended using its in-built scripting language. It also has document management features to help with large documents. As already mentioned, it has styles that can ensure manageable and consistent presentation. Yet very few people seem to take advantage of them. This is especially worsened by UI improvements that mean Word will hide features that you do not use, which makes it more difficult to remember what Word can actually do. \label{ui}
  \item Quite simply, anyone who is writing non-trivial documents and is tired of being let down by the performance of the current crop of word processors. If you are in academia, you really ought to be using it! Anybody writing anything maths related will not find a richer and better quality system. For example, even WikiPedia use LaTeX for rendering any formulas that appear on their site.
  \item LaTeX isn't for people who are too lazy or dislike change! I personally found the investment paid off because LaTeX allows me to produce my documents at a greater pace. I know that the enterprise will not be interested as Word is so ingrained, even though their business reports would look so much nicer. Their loss! For everyone else, it's time to give it a fair try, just so that you compare and contrast, then decide which does the job best for your needs.
\end{enumerate}

Regarding \ref{ui}, verify this last sentence.

%             1. Unique features
%                1. Math typesetting
%                2. Programmatic code highlighting
%                3. Labels and references

Quoted from https://tug.org/texshowcase/:
\begin{enumerate}
  \item You can compare a Word Processor (e.g. MS Word) setup to a TeX setup as a Camper (or RV) versus having a house and a car. The Camper is for everything: you can live in it, you can drive with it and you can look at it. The Word Processor is like a Camper: it does editing, formatting/typesetting, and displaying. It is not excellent at any of these functions, but the combination is pretty neat. In a TeX setup, these functions are separated, like with having a house and a car.
  \item Some things are, however, difficult to do in TeX. Mostly these are the kind of things where you want very fine-grained control over exact positioning of images, wrapping around these images, etc. You can do this in TeX, but it is often (very) cumbersome to get it right and changes may be a lot of work.\\
\end{enumerate}

Quoted from https://tex.stackexchange.com/questions/94889/how-can-i-explain-the-meaning-of-latex-to-my-grandma:
\begin{enumerate}
  \item LaTeX is to a book what a set of blueprints is to a building.
  \item The LaTeX user is the architect that, designs the blueprint, for the computer and printer to build... Any trade-y can throw together a shed without any kinda of plan, but a beautiful building requires blueprints
  \item It does this [movable type] but it uses a computer and so requires less manual labour.
  \item https://tex.stackexchange.com/a/94910 (has photos)\\
\end{enumerate}

Quoted from


\vspace{2cm}

First cover features, operation (TUI: writing code and compiling), and general workflow. Cover the latex mindset.

\subsubsection{Don't Switch to \LaTeX\ If...}%
\label{feat-comp.latex.don't-switch}

Don't switch to LaTeX if

1. You're not willing to learn the minimum necessary for your needs (might just be math equations, or might extend to bibliographies).

2. All this being said, LaTeX is just text (another benefit!). If you've started a project but can't finish it in LaTeX, you can always copy/paste it somewhere else, or even export it directly to docx with tools like Pandoc (won't export everything perfectly).

\subsection{Table of Features}%
\label{feat-comp.table}

We should quantify as many features as possible and present them in a table for a comprehensive feature comparison.

\subsubsection{\LaTeX\ vs. General WP}%
\label{feat-comp.table.latex-vs-wp}

\subsubsection{Overleaf vs. Google Drive vs. Microsoft Onedrive}%
\label{feat-comp.table.collab-solns}

\section{How Tos}%
\label{how-to}

\subsection{How to Read Error Messages}%
\label{how-to.error-messages}

\subsection{How to Debug \LaTeX\ Code}%
\label{how-to.debug}

\subsection{How to Use \LaTeX\ Package Documentation}%
\label{how-to.package-docs}

\subsection{How to Find Help}%
\label{how-to.help}

tex stackexchanger, overleaf.com/learn, google searches

\subsection{How to Use Detexify}%
\label{how-to.detexify}

Really simple.

\section{Building Blocks}%
\label{bbs}

\subsection{Preambles}%
\label{bbs.preambles}

\subsubsection{Useful Packages}%
\label{bbs.preambles.packages}

\subsubsection{Homework}%
\label{bbs.preambles.homework}

\subsubsection{Reports}%
\label{bbs.preambles.reports}

\paragraph{IQP}%
\label{bbs.preambles.reports.iqp}

\paragraph{MQP}%
\label{bbs.preambles.reports.mqp}

\subsection{Tweaks}%
\label{bbs.tweaks}

We can add the code snippet then link some documentation for people to dive deeper.

\begin{enumerate}
  \item Header and footer for nice homework.
  \item Two figures side by side.
  \item Appendices
  \item Bibliography
  \item acronyms
  \item landscape pages
  \item block quotes
  \item list spacing
  \item actually put floats where you want them
  \item different fonts and math fonts
  \item footnotes
  \item glossaries
  \item page margins
  \item date formatting
  \item multi-row table cells
  \item code styling
  \item include PDF pages
  \item charts and other graphs
  \item justification
  \item line spacing
  \item reduce vertical space between word and underline
  \item vertical table cells
  \item URL colors
  \item single space table of contents
\end{enumerate}

\subsection{Snippet Plugins}%
\label{bbs.snippet-plugins}

Vim, emacs, vscode.

Naming convetions

Good to have: fraction, etc.

\subsection{Naming Convetions}%
\label{bbs.name-convs}

\subsubsection{Sections}%
\label{bbs.name-convs.sections}

Good practices for naming sections

\subsubsection{Labels}%
\label{bbs.name-convs.labels}

Good practices for naming labels, e.g. \texttt{sec.subsec.subsubsec} (dot notation) or \texttt{sec:subsec::subsubsec} (colon notation).

\subsection{How To Manually Modifying Spacing}%
\label{bbs.spacing}

How to use positive and negative vspace and hspace. How to use vfill and hfill.

Macros for inserting a preset amount of horizontal space, e.g. to separate answers within a box

\subsection{\LaTeX\ Editing Workflow}%
\label{workflow}

Your own preamble. File templates. Homework templates. Bibliographies

\section{\LaTeX\ Editors}%
\label{latex-editors}

TUI (emacs and vim), VSCode, overleaf. Find other popular ones to cover.

\section{Overleaf for Collaborative Editing}%
\label{overleaf}

Don't give Overleaf tutorial (they have docs for that). We should outline how to manage group projects and collaborative writing in latex.

\section{Inkscape and \LaTeX\ For Fancy Diagrams}%
\label{inkscape-latex}

\end{document}
